\documentclass{article}

\usepackage{physics}
\usepackage{parskip}
\usepackage{listings}
\usepackage{graphicx}
\usepackage{xcolor}
\usepackage{enumerate}
\usepackage[margin=2cm]{geometry}
\usepackage{subfig}
\usepackage{caption}
\usepackage{siunitx}
\usepackage{booktabs}

\renewcommand{\vec}{\mathbf}
\renewcommand{\u}{\vec{v}}

\renewcommand{\Re}{\text{Re}}

\renewcommand{\maketitle}[1]{
\begin{center}
{\Large
TEP4165 2023 \\
Øving #1 \\
\normalsize
Magnus Lilledahl
}
\end{center}
}

\definecolor{codegreen}{rgb}{0,0.6,0}
\definecolor{codegray}{rgb}{0.5,0.5,0.5}
\definecolor{codepurple}{rgb}{0.58,0,0.82}
\definecolor{backcolour}{rgb}{0.95,0.95,0.92}

\lstdefinestyle{mystyle}{
    backgroundcolor=\color{backcolour},   
    commentstyle=\color{codegreen},
    keywordstyle=\color{magenta},
    numberstyle=\tiny\color{codegray},
    stringstyle=\color{codepurple},
    basicstyle=\ttfamily\footnotesize,
    breakatwhitespace=false,         
    breaklines=true,                 
    captionpos=b,                    
    keepspaces=true,                 
    numbers=left,                    
    numbersep=5pt,                  
    showspaces=false,                
    showstringspaces=false,
    showtabs=false,                  
    tabsize=2
}

\lstset{style=mystyle}

\begin{document}

\maketitle{10}

\section{Task 1}

\subsection{a)}

The 2D energy equation for a convective flow including heat conduction is

\begin{equation}
\label{eq:energy2d}
\partial_x{(\rho c_p T u)} + \partial_y{(\rho c_p T v)} = \partial_x(k\partial_x T) + \partial_y(k\partial_y T)
\end{equation}

We assume that $v=0$, i.e. only convection in the $u$-direction. We integrate the (\ref{eq:energy2d}) over a rectangular control volume with edges along coordinate axes and unit length in the $z$-direction.

\begin{equation}
\Delta y \left[ (\rho c_p Tu)_e - (\rho c_p Tu)_w \right]= \Delta y \left[(k\partial_x T)_e - (k \partial_x T)_w \right] + \Delta x \left[(k\partial_y T)_n - (k \partial_y T)_s \right]
\end{equation}

We approximate face values of $T$ by linear interpolation and derivatives by central differences to yield

\begin{equation}
F_e \frac{T_E + T_P}{2} - F_w \frac{T_P + T_W}{2} = D_e(T_E-T_P) - D_w(T_P-T_W) + D_n(T_N-T_P) - D_s(T_P-T_S).
\end{equation}

Where we have introduce $ F = \rho u A$ and $ D = \frac{k A}{c_p \delta x}$. Note $A = \Delta x \cdot 1$ and similarly for $\Delta y$. Now group terms with $T_i$

\begin{equation}
(F_e - F_w + D_e + D_w + D_n + D_s) T_P = (D_e - F_e) T_E + (D_w+F_w) T_w + D_n T_N - D_s T_P.
\end{equation}

The continutity equation is 

\begin{equation}
\partial_x \rho u = 0
\end{equation}

Integrating over a control volume we get

\begin{equation}
(\rho u)_e - (\rho u)_w  = F_e - F_w = 0
\end{equation}

We can therefore simplify to

\begin{equation}
(D_e + D_w + D_n + D_s) T_P = (D_e - F_e) T_E + (D_w+F_w) T_w + D_n T_N - D_s T_P.
\end{equation}

This holds for the interior points.

\subsection{b)}

For the left boundary we assume fixed temperature that extends to the left, thus the above equation can be used by just replacing $T_W = T_a$. This would place the boundary at $-\Delta x$. Alternatively we can just used the western flux and modifiy the delta x at the boundary.

We will assume that at the right boundary we have von Neuman boundary conditions, i.e. no temperature gradient in the y-direction, indicating the the flow has reached a steady state. This can be checke by changing the length of the simulation domain and ensuring that temperature does not change at the initial right boundary.

\subsection{c}

Extend Fortran code

\end{document}
