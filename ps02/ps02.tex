\documentclass{article}

\usepackage{physics}
\usepackage{parskip}
\usepackage{listings}
\usepackage{graphicx}
\usepackage{xcolor}
\usepackage{enumerate}
\usepackage[margin=2cm]{geometry}
\usepackage{subfig}
\usepackage{caption}
\usepackage{siunitx}
\usepackage{booktabs}

\renewcommand{\vec}{\mathbf}
\renewcommand{\u}{\vec{v}}

\renewcommand{\Re}{\text{Re}}

\renewcommand{\maketitle}[1]{
\begin{center}
{\Large
TEP4165 2023 \\
Øving #1 \\
\normalsize
Magnus Lilledahl
}
\end{center}
}

\definecolor{codegreen}{rgb}{0,0.6,0}
\definecolor{codegray}{rgb}{0.5,0.5,0.5}
\definecolor{codepurple}{rgb}{0.58,0,0.82}
\definecolor{backcolour}{rgb}{0.95,0.95,0.92}

\lstdefinestyle{mystyle}{
    backgroundcolor=\color{backcolour},   
    commentstyle=\color{codegreen},
    keywordstyle=\color{magenta},
    numberstyle=\tiny\color{codegray},
    stringstyle=\color{codepurple},
    basicstyle=\ttfamily\footnotesize,
    breakatwhitespace=false,         
    breaklines=true,                 
    captionpos=b,                    
    keepspaces=true,                 
    numbers=left,                    
    numbersep=5pt,                  
    showspaces=false,                
    showstringspaces=false,
    showtabs=false,                  
    tabsize=2
}

\lstset{style=mystyle}

\begin{document}



\section{Task 1}

\subsection*{a)}

\textbf{Discretization}
%Discretize heat with standard FVM and explicit Euler

Starting with the heat equation

\begin{equation}
	\partial_t T = \alpha \partial_x^2 T
\end{equation}

we write this as an integral equation on the interval $x_w$, $x_e$.

\begin{equation}
	\int_{x_w}^{x_e}  \partial_t T \dd{x} = \int_{x_w}^{x_e}  \alpha \partial_x^2 T \dd{x}
\end{equation}

On the LHS take the time derivative outside the integral, approximate the integral by the value at the midpoint multiplied by the interval $T_P \Delta x$, and assume $\alpha$ constant to give

\begin{equation}
 	\partial_t T_P = \alpha (\partial_x T |_{x_w} - \partial_x T |_{x_e})
\end{equation}

Use explicit euler to discretize the time derivative and a central schemer for the space derivatives resulting in

\begin{equation}
 	T_P^{n+1} = T_P^n + r ( -2T_P^n + T_E^n + T_W^n)
\end{equation}

where $r = \dfrac{\alpha \Delta t }{\Delta x^2} $. For the boundarys we have 

\begin{equation}
 	T_{NJ}^{n+1} = T_{NJ}^n + r ( -3T_{NJ}^n + T_{NJ-1}^n + 2T_2^n)
\end{equation}

\begin{equation}
 	T_{1}^{n+1} = T_1^n + r ( -3T_{1}^n + T_{2}^n + 2T_1^n)
\end{equation}




\textbf{Convergence}

We check first for consistency.

xxx

%Show convergent.
To show that convergence we will assume that $D$ is the exact solution to the discretized equation and $N = D+\epsilon$ the numerical solution. Since the discretized equation is linear it also satisfied by the numerical error $\epsilon$. Assume that the solution can be written as a Fourier sum $\epsilon = \sum_n\sum_n b e^{ik_nx}e^{ia_mt}$. If we focus on a single frequency component $a = a_m$  we can than see that a subsequent time step can be written as $\epsilon^{n+1}=e^{ia\Delta t}\epsilon^n$. We will assume the method is stable if 

\begin{equation}	
	|G| = |e^{ia\Delta t}\| \leq 1
\end{equation}

If we plug a single component of the numerical error into the discretized equation we end up with

\begin{equation}
 	e^{ia\Delta t} = 1 + r(-2+e^{ik\Delta x}-e^{-ik\Delta x}) = 1 - 2r(1-\cos(\beta))
\end{equation}

with $\beta = k \Delta x$.

We thus need

\begin{equation}
	|1 - 2r(1-\cos(\beta))| \leq 1
\end{equation}

Checking first the case LHS inside the absolute value is positive

\begin{equation}
	1 - 2r(1-\cos(\beta)) \leq 1
\end{equation}

which corresponds to 

\begin{equation}
	\cos(\beta) \leq 1
\end{equation}

which is always satisified. Checking the case if LSH inside absolute value is negeative

\begin{equation}
	-(1 - 2r(1-\cos(\beta))) \leq 1
\end{equation}

\begin{equation}
	 2r(1-\cos(\beta)) \leq 2
\end{equation}

Resulting in

\begin{equation}
	r(1-\cos(\beta)) \leq 1
\end{equation}

The maximum value in the parenthesis is 2 (for $\cos\beta = -1)$ so to ensure the inequality is satisfied we must have

\begin{equation}
 	0 \leq \leq \frac{1}{2}
\end{equation}

The greater than zero condition follows from the assumption that the LHS was negative.

Since the system is consistent and stable we say it is convergent (approaching true solution to PDE when grid size is reduced).

\subsection*{b}

For the implicit method

\section*{Task 2}

\subsection*{Task 1}

Solution algorithm

\begin{lstlisting}
# Initialize values

# Create two arrays holding the temperature at the center of the finite elements. 
#One for new and previous time step.

# Loop for each time step
	# Update endpoints with discretization for boundary values.
	# Update interior points with the discretization step in 1a)
	# Update previous time step with new time step
\end{lstlisting}

\end{document}

