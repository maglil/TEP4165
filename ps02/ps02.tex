\documentclass{article}

\usepackage{physics}
\usepackage{parskip}
\usepackage{listings}
\usepackage{graphicx}
\usepackage{xcolor}
\usepackage{enumerate}
\usepackage[margin=2cm]{geometry}
\usepackage{subfig}
\usepackage{caption}
\usepackage{siunitx}
\usepackage{booktabs}

\renewcommand{\vec}{\mathbf}
\renewcommand{\u}{\vec{v}}

\renewcommand{\Re}{\text{Re}}

\renewcommand{\maketitle}[1]{
\begin{center}
{\Large
TEP4165 2023 \\
Øving #1 \\
\normalsize
Magnus Lilledahl
}
\end{center}
}

\definecolor{codegreen}{rgb}{0,0.6,0}
\definecolor{codegray}{rgb}{0.5,0.5,0.5}
\definecolor{codepurple}{rgb}{0.58,0,0.82}
\definecolor{backcolour}{rgb}{0.95,0.95,0.92}

\lstdefinestyle{mystyle}{
    backgroundcolor=\color{backcolour},   
    commentstyle=\color{codegreen},
    keywordstyle=\color{magenta},
    numberstyle=\tiny\color{codegray},
    stringstyle=\color{codepurple},
    basicstyle=\ttfamily\footnotesize,
    breakatwhitespace=false,         
    breaklines=true,                 
    captionpos=b,                    
    keepspaces=true,                 
    numbers=left,                    
    numbersep=5pt,                  
    showspaces=false,                
    showstringspaces=false,
    showtabs=false,                  
    tabsize=2
}

\lstset{style=mystyle}

\begin{document}

\maketitle{2}

\section{Task 1}

\subsection*{a)}

\textbf{Discretization}
%Discretize heat with standard FVM and explicit Euler

Starting with the heat equation

\begin{equation}
	\partial_t T = \alpha \partial_x^2 T
\end{equation}

we write this as an integral equation on the interval $x_w$, $x_e$.

\begin{equation}
	\int_{x_w}^{x_e}  \partial_t T \dd{x} = \int_{x_w}^{x_e}  \alpha \partial_x^2 T \dd{x}
\end{equation}

On the LHS take the time derivative outside the integral, approximate the integral by the value at the midpoint multiplied by the interval $T_P \Delta x$, and assume $\alpha$ constant to give

\begin{equation}
 	\Delta x\partial_t T_P = \alpha (\partial_x T |_{x_w} - \partial_x T |_{x_e})
\end{equation}

Use explicit euler to discretize the time derivative and a central schemer for the space derivatives resulting in

\begin{equation}
	\label{eq:explicit}
 	T_P^{n+1} = T_P^n + r ( -2T_P^n + T_E^n + T_W^n)
\end{equation}

where $r = \dfrac{\alpha \Delta t }{\Delta x^2} $. For the boundaries we have 

\begin{equation}
 	T_{NJ}^{n+1} = T_{NJ}^n + r ( -3T_{NJ}^n + T_{NJ-1}^n + 2T_2^n)
\end{equation}

\begin{equation}
 	T_{1}^{n+1} = T_1^n + r ( -3T_{1}^n + T_{2}^n + 2T_1^n)
\end{equation}




\textbf{Convergence}

We check first for \emph{consistency}.

Taylor expanding each side of (\ref{eq:explicit}) and using $T= T_P$ and subscripts do indicate partial derivatives.

\begin{equation}
 	T + \Delta T_t + O(\Delta t^2) = T + r(-2T + T + \Delta x T_x + \frac{\Delta x^2}{2}Txx + O(\Delta x^3) - ( T - \Delta x T_x + \frac{\Delta x^2}{2}T_{xx} + O(\Delta x^3)
\end{equation}

Cancelling terms we get

\begin{equation}
 	T_t + O(\Delta t) = \alpha T_{xx} + O(\Delta x)
\end{equation}

As $\Delta t \rightarrow 0 $ and $\Delta x \rightarrow 0 $ we see that we get the original PDE so the scheme is \emph{consistent}. We also see it is first order in time and space. 

Second we check for \emph{stability}. We will assume that $D$ is the exact solution to the discretized equation and $N = D+\epsilon$ the numerical solution. Since the discretized equation is linear it also satisfied by the numerical error $\epsilon$. Assume that the error can be written as a Fourier sum $\epsilon = \sum_n\sum_n b e^{ik_nx}e^{ia_mt}$. If we focus on a single frequency component $a = a_m$  we can than see that a subsequent time step can be written as $\epsilon^{n+1}=e^{ia\Delta t}\epsilon^n$. We will assume the method is stable if 

\begin{equation}	
	|G| = |e^{ia\Delta t}\| \leq 1
\end{equation}

That is, errors are not amplified at each time step. If we plug a single component of the numerical error into the discretized equation we end up with

\begin{equation}
 	e^{ia\Delta t} = 1 + r(-2+e^{ik\Delta x}-e^{-ik\Delta x}) = 1 - 2r(1-\cos(\beta))
\end{equation}

with $\beta = k \Delta x$.

We thus need

\begin{equation}
	|1 - 2r(1-\cos(\beta))| \leq 1
\end{equation}

Checking first the case LHS inside the absolute value is positive

\begin{equation}
	1 - 2r(1-\cos(\beta)) \leq 1
\end{equation}

which corresponds to 

\begin{equation}
	\cos(\beta) \leq 1
\end{equation}

which is always satisified. Checking the case if LSH inside absolute value is negeative

\begin{equation}
	-(1 - 2r(1-\cos(\beta))) \leq 1
\end{equation}

\begin{equation}
	 2r(1-\cos(\beta)) \leq 2
\end{equation}

Resulting in

\begin{equation}
	r(1-\cos(\beta)) \leq 1
\end{equation}

The maximum value in the parenthesis is 2 (for $\cos\beta = -1)$ so to ensure the inequality is satisfied we must have

\begin{equation}
 	0 \leq r \leq \frac{1}{2}
\end{equation}

The greater than zero condition follows from the assumption that the LHS was negative.

Since the system is \emph{consistent} and \emph{stable} we say it is \emph{convergent} (approaching true solution to PDE when grid size is reduced).

\subsection*{b)}

For the implicit method the resulting discretization is

\begin{equation}
 	T_P^{n+1} -  T_P^n=  r ( -2T_P^{n+1} + T_E^{n+1} + T_W^{n+1})
\end{equation}

which can be simplified to

\begin{equation}
	\label{eq:implicit}
 	T_P^{n+1}(1+2r) -rT_E^{n+1} -r T_W^{n+1} =  T_P^n
\end{equation}

Taylor expanding about $T_P^{n+1} = T$

\begin{equation}
T(1+2r) - r(T + T_x \Delta x + T_{xx} \frac{\Delta x^2}{2} + O(\Delta x^3)) - r(T - T_x \Delta x + T_xx \frac{\Delta x^2}{2}+ O(\Delta x^3)) = T - \Delta t T_t + O(\Delta t^2)
\end{equation}

Simplifying the expression we end up with

\begin{equation}
 	-r (T_{xx}\Delta x^2 +  O(\Delta x^3) ) = -\Delta t T_t  + O(\Delta t^2)
\end{equation}

Dividing through by $\Delta t$ and using $r = \frac{\alpha \Delta t}{\Delta x^2}$

\begin{equation}
 	\alpha T_{xx} +  O(\Delta x)  = T_t  + O(\Delta t)
\end{equation}

We see this reduces to the original heat equation as the $\Delta$ go to zero and is thus consistent. The method is first order accurate.

\textbf{Stability}

Inserting the fourier components into equation \ref{eq:implicit} as in part 1 yields after dividing through by the fourier component

\begin{equation}
(1+2r) - re^{ik\Delta x} - re^{ik\Delta x} = e^{-i\Delta t}
\end{equation}

Using Eulers identity and $k\Delta x = \beta$

\begin{equation}
 	1+2r - 2r\cos(\beta) =  e^{-i\Delta t}
\end{equation}

or

\begin{equation}
 	G = e^{i\Delta t} = \frac{1}{1+2r - 2r\cos(\beta) }
\end{equation}

For stability we must have $|G| \leq 1$

\begin{equation}
 	|1 + 2r(1-\cos(\beta) )| \geq 1
\end{equation}

The expression inside the absolut value is always positive so we get


\begin{equation}
 	1 + 2r(1-\cos(\beta) ) \geq 1
\end{equation}

which simplifies to 

\begin{equation}
 	\cos(\beta) \leq 1
\end{equation}

which is always true. The method is therefore unconditionally stable, and therefore convergent.

\section*{Task 2}

\subsection*{a)}

Solution algorithm

\begin{lstlisting}
# Initialize values

# Create two arrays holding the temperature at the center of the finite elements. 
# One for new and previous time step.

# Loop for each time step
	# Update endpoints with discretization for boundary values.
	# Update interior points with the discretization step in 1a)
	# Update previous time step with new time step
\end{lstlisting}

\subsection*{b)}

The code for the remainder of task 2 is shown at the end of the document.

\subsection*{c)}

Figure \ref{fig:t01} shows the temperature distribution at $t=0.1$ s. The interior points are at the inital temperature and the boundary values are shown. Seems reasonable.

\begin{figure}
	\includegraphics{t_0.1.png}
	\caption{Temperature distribution at $t=0.1$ s}
	\label{fig:t01}
\end{figure}

\subsection*{d)}
Figure \ref{fig:t40} shows the temperature distribution at $t=40$ s.

\begin{figure}
	\includegraphics{t_40.png}
	\caption{Temperature distribution at $t=40$ s}
	\label{fig:t40}
\end{figure}

\subsection*{e)}
Figure \ref{fig:unstable} shows the simulation result when the stability criterion($r<0.5$) is violoate with  $=0.6$ s.

\begin{figure}
	\includegraphics{unstable.png}
	\caption{Simulation result when the stability criterion is violated ($r=0.6$ s).}
	\label{fig:unstable}
\end{figure}

\subsection*{f)}


Figure \ref{fig:nj50}-\ref{fig:nj400} show the difference between the numerical and analytic solution for different spatial grid spacing.

\begin{figure}[h!]
\begin{minipage}{0.55\textwidth}
            \includegraphics[width=\textwidth]{numerical_analytic50.png}
            \caption{\small NJ=50}
            \label{fig:nj50}
\end{minipage}
\begin{minipage}{0.55\textwidth}
            \includegraphics[width=\textwidth]{numerical_analytic100.png}
            \caption{\small NJ=100}
            \label{fig:nj100}
\end{minipage}
\begin{minipage}{0.55\textwidth}
            \includegraphics[width=\textwidth]{numerical_analytic200.png}
            \caption{\small NJ=200}
            \label{fig:nj200}
\end{minipage}
\begin{minipage}{0.55\textwidth}
            \includegraphics[width=\textwidth]{numerical_analytic400.png}
            \caption{\small NJ=400}
            \label{fig:nj400}
\end{minipage}
\end{figure}

\subsection*{g)}
The computed errors were 0.12, 0.061, 0.030, 0.015.

\subsection*{h)}
The convergence rate was computed to be about 1 which is compatible of the consistency analysis which indicated a first order method (Computed values for each of the pairs of grid spacings, 0.996, 1.0132, 1.0145).

\clearpage
\newpage

\lstinputlisting[language = matlab]{ps02.m}

\end{document}

