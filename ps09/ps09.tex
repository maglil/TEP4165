\documentclass{article}

\usepackage{physics}
\usepackage{parskip}
\usepackage{listings}
\usepackage{graphicx}
\usepackage{xcolor}
\usepackage{enumerate}
\usepackage[margin=2cm]{geometry}
\usepackage{subfig}
\usepackage{caption}
\usepackage{siunitx}
\usepackage{booktabs}

\renewcommand{\vec}{\mathbf}
\renewcommand{\u}{\vec{v}}

\renewcommand{\Re}{\text{Re}}

\renewcommand{\maketitle}[1]{
\begin{center}
{\Large
TEP4165 2023 \\
Øving #1 \\
\normalsize
Magnus Lilledahl
}
\end{center}
}

\definecolor{codegreen}{rgb}{0,0.6,0}
\definecolor{codegray}{rgb}{0.5,0.5,0.5}
\definecolor{codepurple}{rgb}{0.58,0,0.82}
\definecolor{backcolour}{rgb}{0.95,0.95,0.92}

\lstdefinestyle{mystyle}{
    backgroundcolor=\color{backcolour},   
    commentstyle=\color{codegreen},
    keywordstyle=\color{magenta},
    numberstyle=\tiny\color{codegray},
    stringstyle=\color{codepurple},
    basicstyle=\ttfamily\footnotesize,
    breakatwhitespace=false,         
    breaklines=true,                 
    captionpos=b,                    
    keepspaces=true,                 
    numbers=left,                    
    numbersep=5pt,                  
    showspaces=false,                
    showstringspaces=false,
    showtabs=false,                  
    tabsize=2
}

\lstset{style=mystyle}

\begin{document}
\maketitle{9}

\section{Task 1}

\subsection{a)}
% Discretize heat eq using FVM

We start with the 1D heat equation

\begin{equation}
0 = \partial_x (k \partial_x T) + \partial_y( k \partial_y  T)
\end{equation}

Integrate this over a rectangular control volume, assuming no variation in the $z$-direction results in

\begin{equation}
\label{eq:fvm}
0 = (k \partial_x T)_e A_e - (k \partial_x T)_w A_w + (k \partial_x T)_n A_n - (k \partial_x T)_e A_s
\end{equation}

Where $(k \partial_x T)_e = \frac{1}{A_e}\int_{\partial \Omega_e} k \partial_x T dA$, i.e. the average value over the eastern surface. Similarly for the other sides.

The thermal conductivites and temperature derivatives are then approximated by cell center values. We use a central difference for the temperature to give.

\begin{equation}
0=k_e \frac{T_E - T_P}{\delta x_{PE} } A_e - k_w \frac{T_P - T_W}{\delta x_{WP} } A_w + k_n \frac{T_N - T_P}{\delta x_{PN} } A_n - k_s \frac{T_P - T_S}{\delta x_{SP} } A_s
\end{equation}

Now we group terms $T_i$ which results in

\begin{equation}
\left( \frac{k_e A_e}{\delta x_{PE}} + \frac{k_w A_w}{\delta x_{WP}} + \frac{k_n A_n}{\delta x_{PN}}  + \frac{k_s A_s}{\delta x_{SP}} \right) T_P = \frac{k_e A_e}{\delta x_{PE}} T_E + \frac{k_w A_w}{\delta x_{WP}} T_W + \frac{k_n A_n}{\delta x_{PN}} T_N + \frac{k_s A_s}{\delta x_{SP}} T_S
\end{equation}

which we then rewrite as

\begin{equation}
\label{eq:patankar}
a_p T_P = a_E T_E + a_W T_W + a_N T_N + a_S T_S
\end{equation}


\subsection{b)}

% Write fortran program. Gauss seidel. extend previous program
The fortran program that solves the 2D heat equation is shown at the end of document. For an inital guess of constant temperature $T_0 = 450$, 6000 iterations were needed for convergence (for $T_0 = 0$, about 10000 iterations were needed).

\subsection{c)}

% Plot T countours and comment
Figure \ref{fig:tdist} shows the temperature distribution. The temperature is as expected a gradual slope from the high to the low temperatures. 

\begin{figure}
\includegraphics[width= 0.7\textwidth]{ps09cT.png}
\caption{Temperature distribution}
\label{fig:tdist}
\end{figure}

\subsection{d)}
% Change boundary conditions to heat flux
When the boundary conditions change to a prescribed heat flux (equivalent to prescribed gradient of the temperature, i.e. von Neumann boundary conditions). The basic equation (\ref{eq:fvm}) in a) is the same for interior points and points next to a boundary with fixed temperatur (Dirichlet boundary condition). However, for boundary cells with prescribed heat flux through a surface, the appropriate flux term in (\ref{eq:fvm}) is replaced by the value given by the boundary condition. E.g. for our problem with a flux at the eastern boundary we get

\begin{equation}
\label{eq:fvm_vn}
0 = q - (k \partial_x T)_w A_w + (k \partial_x T)_n A_n - (k \partial_x T)_e A_s
\end{equation}

We can rewrite this as (\ref{eq:patankar}) above

\begin{equation}
\label{eq:patankar}
a_p T_P = a_E T_E + a_W T_W + a_N T_N + a_S T_S + S_u
\end{equation}

where for the cells next to the eastern boundary, $a_E = 0$ and $S_u = q$. The western and southern boundary nodes are treated similarly with $a_W = 0$ and $a_E= 0$, respectively and $q=0$

\subsection{e)}
In the fortran code a parameter \texttt{sys} is added to switch between the different boundary conditions. The code is shown at the end of the document. The inital temperature was $T = 400$ and convergence took about 8000 iterations.

\subsection{f)}

% Plot temperature contours
 The temperature distribution with the given heat flux is shown in fig. \ref{fig:3}. The temperature becomes very high so I suspect and error. I reduced the flux by a factor of 100 which resultet in the plot in fig.\ref{fig:emod} where the other boundary conditions are more apparent.

\begin{figure}
\includegraphics[width= 0.7\textwidth]{ps09eT.png}
\caption{Temperature distribution}
\label{fig:e}
\end{figure}

\begin{figure}
\includegraphics[width= 0.7\textwidth]{ps09eTmod.png}
\caption{Temperature distribution}
\label{fig:emod}
\end{figure}

\subsection{g)}

% Grid refinement

\subsection{h)}

Alternating line Gauss-Seidel

\subsection{i)}

Multigrid method

\lstinputlisting[language=fortran]{heat2d.f90}







\end{document}