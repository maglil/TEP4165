\documentclass{article}

\usepackage{physics}
\usepackage{parskip}
\usepackage{listings}
\usepackage{graphicx}
\usepackage{xcolor}
\usepackage{enumerate}
\usepackage[margin=2cm]{geometry}
\usepackage{subfig}
\usepackage{caption}
\usepackage{siunitx}
\usepackage{booktabs}

\renewcommand{\vec}{\mathbf}
\renewcommand{\u}{\vec{v}}

\renewcommand{\Re}{\text{Re}}

\renewcommand{\maketitle}[1]{
\begin{center}
{\Large
TEP4165 2023 \\
Øving #1 \\
\normalsize
Magnus Lilledahl
}
\end{center}
}

\definecolor{codegreen}{rgb}{0,0.6,0}
\definecolor{codegray}{rgb}{0.5,0.5,0.5}
\definecolor{codepurple}{rgb}{0.58,0,0.82}
\definecolor{backcolour}{rgb}{0.95,0.95,0.92}

\lstdefinestyle{mystyle}{
    backgroundcolor=\color{backcolour},   
    commentstyle=\color{codegreen},
    keywordstyle=\color{magenta},
    numberstyle=\tiny\color{codegray},
    stringstyle=\color{codepurple},
    basicstyle=\ttfamily\footnotesize,
    breakatwhitespace=false,         
    breaklines=true,                 
    captionpos=b,                    
    keepspaces=true,                 
    numbers=left,                    
    numbersep=5pt,                  
    showspaces=false,                
    showstringspaces=false,
    showtabs=false,                  
    tabsize=2
}

\lstset{style=mystyle}

\begin{document}

\maketitle{11}

\section{Task 1}

\subsection{a)}
% Discretize momentum and energy equations. Upwind. Staggered grid notation.

For the x-momentum equation, we consider a control volume that is centered on a node $i,J$

We first conside the conective term

\begin{equation}
\int\partial_x (\rho u^2) dV \approx (\rho u^2 A)_e - (\rho u^2 A)_w
\end{equation}

We linearize the equation by introducing the convective flux

\begin{equation}
F = \rho u A
\end{equation}

So that 

\begin{equation}
(\rho u^2 A)_e - (\rho u^2 A)_w = F_e u_e - F_w u_w
\end{equation}

The convective flux is approximated by a central average

\begin{equation}
F_e  = \oh (F_E + F_P)
\end{equation}

While $F_e u_e$ is approximated by the upwind method

\begin{align}
F_e u_e &= \oh [F_e u_P + F_e u_E  + |F_e| (u_E - u_P) ]\\
	&= \oh [(F_e + |F_e|)u_P + (F_e - |F_e|)u_E \\
	&= F_e^+ u_P + F_e^- u_E
\end{align}

We get similarly for $F_wu_w$

\begin{equation}
	F_wu_w = F_w^+ u_w + F_w^- u_P
\end{equation}

In summary for the convective term we get

\begin{equation}
F_e^+ u_P + F_e^- u_E -  F_w^+ u_w + F_w^- u_P
\end{equation}

For the cross term we solv it similaryl on the southern and northern faces to end up with

\begin{equation}
F_n^+ u_P + F_n^- u_N -  F_s^+ u_s + F_s^- u_P
\end{equation}

Moving on to the diffusive term

\begin{equation}
\int \partial_x \mu \partial_x u = (\mu \partial_x u)_e - (\mu \partial_x u)_w
\end{equation}

We introduce the diffusive conductance $D$

\begin{equation}
D = \frac{\mu A}{\delta x}
\end{equation}

So that we get, using a central approximation for derivative

\begin{equation}
D_e (u_E - u_P) - D_w (u_P - u_W)
\end{equation}

Any similarly for the y-diffsuions

Finally for the pressure term we get

\begin{equation}
\int (-\partial_x) dV = -((pA)_e -(pA)_w) = S_u
\end{equation}

The (current) pressure is given at the faces (nodes of the scalar grid) so this can be assigned as a source term.

Collecting all terms we get

\begin{equation}
a_p \phi_P = a_N \phi_E + ...
\end{equation}•

\begin{align}
\phi_P (F_e^+ + F_w^- + F_n^+ + F_s^- - (D_e +D_w +D_n +D_s)
\end{align}

For the y-momentum equation we get similarly

For scalar transport equations we get.

\begin{equation}
\int \rho T u = (ptu)_e - (\rho T u)_w A
\end{equation}

\begin{equation}
F_eT_e - T_w T_w = source?
\end{equation}•



\subsection{b)}
% Complete discretization with staggered grid for pressure.

\subsection{c)}
% Discretize boundary conditions

In the Fortran code the grid is defined with scalar nodes with 2-npi-1 for interior nodes and 1 and npi for the boundary. The velocity grid thus starts at two (first being same and with right boundary at npi. There are npi-2 interior nodes for the scalar values and npi-3 for velocity

\subsection{d)}
% Fix fortran file

The procedure is 
\begin{itemize}
\item Guess pressure.
\item Solve momentum equation iteratively
\item Write momentum equation for correction values
\item Simplify
\item Write conservation of mass with preliminary values at some points and correction at others.
\item Solve
\item repeat

\end{itemize}

\subsection{e}

\subsection{f)}
Convergence check


\end{document}