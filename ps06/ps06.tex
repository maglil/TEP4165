\documentclass{article}

\usepackage{physics}
\usepackage{parskip}
\usepackage{listings}
\usepackage{graphicx}
\usepackage{xcolor}
\usepackage{enumerate}
\usepackage[margin=2cm]{geometry}
\usepackage{subfig}
\usepackage{caption}
\usepackage{siunitx}
\usepackage{booktabs}

\renewcommand{\vec}{\mathbf}
\renewcommand{\u}{\vec{v}}

\renewcommand{\Re}{\text{Re}}

\renewcommand{\maketitle}[1]{
\begin{center}
{\Large
TEP4165 2023 \\
Øving #1 \\
\normalsize
Magnus Lilledahl
}
\end{center}
}

\definecolor{codegreen}{rgb}{0,0.6,0}
\definecolor{codegray}{rgb}{0.5,0.5,0.5}
\definecolor{codepurple}{rgb}{0.58,0,0.82}
\definecolor{backcolour}{rgb}{0.95,0.95,0.92}

\lstdefinestyle{mystyle}{
    backgroundcolor=\color{backcolour},   
    commentstyle=\color{codegreen},
    keywordstyle=\color{magenta},
    numberstyle=\tiny\color{codegray},
    stringstyle=\color{codepurple},
    basicstyle=\ttfamily\footnotesize,
    breakatwhitespace=false,         
    breaklines=true,                 
    captionpos=b,                    
    keepspaces=true,                 
    numbers=left,                    
    numbersep=5pt,                  
    showspaces=false,                
    showstringspaces=false,
    showtabs=false,                  
    tabsize=2
}

\lstset{style=mystyle}

\newcommand{\tref}{\frac{L}{\sqrt{\frac{p_l}{\rho_l}}}}
\newcommand{\vref}{ \sqrt{\frac{p_l}{\rho_l}} }
\renewcommand{\Re}{\text{Re}}

\begin{document}


\maketitle{6}

\section{Task 1}

\subsection{a)}
The nondimensnional variables are

\begin{align}
	x^* &= \frac{x}{L} \\
	t^* &= \frac{t}{L/\sqrt{\frac{p_l}{\rho_l}}} = \frac{t}{t_r} \\
	p^* &= \frac{p}{p_l} \\
	u^* &= \frac{v}{\sqrt{\frac{p_l}{\rho_l}}} = \frac{u}{u_r} \\
	(\rho E)^* &= \frac{\rho E}{p_l} \\
	\rho^* &= \frac{\rho}{\rho_l} \\
	T^* &= \frac{T}{T_l}
\end{align}

\subsection{b)}
The Nondimensionalized operators are

\begin{align}
	\partial_x^* &= L\partial_x \\
	\partial_t^* &= L/\sqrt{\frac{p_l}{\rho_l}} \partial_t = t_r \partial_t\\
	(\partial_x^2)^* &= L^2 \partial_x^2
\end{align}

\subsection{c)}
Nondimensionalize equations

The nondimensionalized continuity equations becomes

\begin{equation}
 	\rho_l \frac{\vref}{ L} (\partial_t)^* \rho^* + \frac{1}{L} \partial_{x^*} (\rho_l\rho^* \vref u^*) = 0
\end{equation}

which simplifies to (skipping the stars)

\begin{equation}
 	\partial_t\rho + \partial_x(\rho u) = 0
\end{equation}

For the momentum equation we get

\begin{equation}
\frac{u_r}{L}\partial_{t^*}(\rho_l\rho^* u_r u^*) + \frac{1}{L}(\rho_l \rho^* u_r^2 (u^*)^2 + p_l p^*)_x = \frac{1}{L^2}(\frac{4}{3}\mu u_r u^*)_{xx}
\end{equation}


Multiply whole equation by $\dfrac{L}{\rho_l u_r^2}$

\begin{equation}
\partial_{t^*}(\rho^*  u^*) + ( \rho^*  (u^*)^2 + \dfrac{1}{\rho_l u_r^2}p_l p^*)_x = \dfrac{1}{L \rho_l u_r}(\frac{4}{3}\mu u^*)_{xx}
\end{equation}

From part a we see that $\dfrac{1}{\rho_l u_r^2}p_l=1$ and $ \dfrac{\mu}{L \rho_l u_r} = \frac{1}{Re}$  so that finally (skipping stars) we get

\begin{equation}
\partial_{t}(\rho  u) + ( \rho  u^2 +  p)_x = \frac{4}{3 \Re}( u)_{xx}
\end{equation}

For the energy equation we get

\begin{equation}
\frac{p_l}{t_r} ((\rho E)*)_{t^*} + \frac{1}{L} ( (p_l(pE)^*+p_l p^*)u_r u^*  )_{x^*} = \frac{1}{L^2} (\frac{2}{3}\mu u_r^2 u^2 + k T_l T)_{xx}
\end{equation}

Dividing through by $\frac{p_l u_r}{L}$ and using that $\frac{p_l}{t_r} = \frac{p_l}{L/u_r}$ we get


\begin{equation}
 ((\rho E)*)_{t^*} +  ( ((pE)^*+p^*)u^*  )_{x^*} = \frac{L}{p_l u_r} \frac{1}{L^2} (\frac{2}{3}\mu u_r^2 u^2 + k T_l T)_{xx}
\end{equation}

Using from part a) that $p_l = \rho u_r^2$ , $ \dfrac{\mu}{L \rho_l u_r} = \frac{1}{Re}$ for the first term in the diffusive flux. For the second term we get

\begin{equation}
\frac{kT_l}{p_l u_r L} =  \frac{k p_l/(\rho_l R}{p_l u_r L} = \frac{k}{\rho_l u_r L R} = \frac{k}{\Re \mu R } = \frac{k\gamma}{\mu (\gamma -1) c_p} = \frac{\gamma}{\Re Pr (\gamma -1)}
\end{equation}

where we have used $ R = \gamma/(\gamma-1) c_p$ and $1/Pr = \frac{k}{c_p \mu}$.

Thus finally we get (removing stars) 

\begin{equation}
 ((\rho E))_{t} +  ( ((pE)+p)u  )_{x} = \frac{1}{\Re} (\frac{2}{3}\mu  u^2 + \frac{\gamma}{Pr (\gamma-1)} T)_{xx}
\end{equation}

\subsection{d)}
Discribe discretization of equations EE and CS

\subsection{e)}
How to handle boundary conditions

\subsection{f)}
Write a script that shows shock profile, test

\subsection{g)}
Compute at t = 100

\subsection{h)}
Test the two-norm of the density changes.


\end{document}
