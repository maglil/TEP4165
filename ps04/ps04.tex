\documentclass{article}

\usepackage{physics}
\usepackage{parskip}
\usepackage{listings}
\usepackage{graphicx}
\usepackage{xcolor}
\usepackage{enumerate}
\usepackage[margin=2cm]{geometry}
\usepackage{subfig}
\usepackage{caption}
\usepackage{siunitx}
\usepackage{booktabs}

\renewcommand{\vec}{\mathbf}
\renewcommand{\u}{\vec{v}}

\renewcommand{\Re}{\text{Re}}

\renewcommand{\maketitle}[1]{
\begin{center}
{\Large
TEP4165 2023 \\
Øving #1 \\
\normalsize
Magnus Lilledahl
}
\end{center}
}

\definecolor{codegreen}{rgb}{0,0.6,0}
\definecolor{codegray}{rgb}{0.5,0.5,0.5}
\definecolor{codepurple}{rgb}{0.58,0,0.82}
\definecolor{backcolour}{rgb}{0.95,0.95,0.92}

\lstdefinestyle{mystyle}{
    backgroundcolor=\color{backcolour},   
    commentstyle=\color{codegreen},
    keywordstyle=\color{magenta},
    numberstyle=\tiny\color{codegray},
    stringstyle=\color{codepurple},
    basicstyle=\ttfamily\footnotesize,
    breakatwhitespace=false,         
    breaklines=true,                 
    captionpos=b,                    
    keepspaces=true,                 
    numbers=left,                    
    numbersep=5pt,                  
    showspaces=false,                
    showstringspaces=false,
    showtabs=false,                  
    tabsize=2
}

\lstset{style=mystyle}

\begin{document}

\section{Task 1}

\maketitle{4}

\subsection{a)}

\begin{lstlisting}{language=matlab}
% Solving 1D Burgers equation with
% Initalize two vectors for new and current values of u
% Set current value to inital condition
% While t < tend
	% For j = 1 to NJ-1
		% Calculate convective flux using upwind values
			% Method to choose correct upwind direction
		% Calculate diffusive flux using central difference
		% Sum all fluxes
		% Calculate new u using explicit euler
	% Set u(NJ) = u(1), periodic boundary condition
	% Increase t by dt
\end{lstlisting}

\subsection{b)}
MUSCL approach

Calculate left and right u values
Use in minmod

\subsection{c)}
The advantage of the MUSCL approach is that it is higher order and has less numerical dissipation while maintaining TVD.

\section{Task 2}

\subsection{a)}

\lstinputlisting[language=matlab]{upwind.m}

\subsection{b)}
\begin{figure}
	\includegraphics[widht=0.8\linewidth]{ps04-2b.png}
\end{figure}

\subsection{c)}
\begin{figure}
	\includegraphics[widht=0.8\linewidth]{ps04-2c.png}
\end{figure}

\subsection{d)}
Compare solutions

\subsection{e}
One way to check the solution is to check how the solution is dependent on the grid spacing.

\subsection{f)}
Grid refinement



\end{document}